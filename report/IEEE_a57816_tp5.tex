
% !TEX encoding = UTF-8 Unicode


%% bare_conf_compsoc.tex
%% V1.4b
%% 2015/08/26
%% by Michael Shell
%% See:
%% http://www.michaelshell.org/
%% for current contact information.
%%
%% This is a skeleton file demonstrating the use of IEEEtran.cls
%% (requires IEEEtran.cls version 1.8b or later) with an IEEE Computer
%% Society conference paper.
%%
%% Support sites:
%% http://www.michaelshell.org/tex/ieeetran/
%% http://www.ctan.org/pkg/ieeetran
%% and
%% http://www.ieee.org/

%%*************************************************************************
%% Legal Notice:
%% This code is offered as-is without any warranty either expressed or
%% implied; without even the implied warranty of MERCHANTABILITY or
%% FITNESS FOR A PARTICULAR PURPOSE! 
%% User assumes all risk.
%% In no event shall the IEEE or any contributor to this code be liable for
%% any damages or losses, including, but not limited to, incidental,
  %% consequential, or any other damages, resulting from the use or misuse
  %% of any information contained here.
  %%
  %% All comments are the opinions of their respective authors and are not
  %% necessarily endorsed by the IEEE.
  %%
%% This work is distributed under the LaTeX Project Public License (LPPL)
  %% ( http://www.latex-project.org/ ) version 1.3, and may be freely used,
  %% distributed and modified. A copy of the LPPL, version 1.3, is included
  %% in the base LaTeX documentation of all distributions of LaTeX released
  %% 2003/12/01 or later.
  %% Retain all contribution notices and credits.
  %% ** Modified files should be clearly indicated as such, including  **
  %% ** renaming them and changing author support contact information. **
  %%*************************************************************************


  % *** Authors should verify (and, if needed, correct) their LaTeX system  ***
  % *** with the testflow diagnostic prior to trusting their LaTeX platform ***
  % *** with production work. The IEEE's font choices and paper sizes can   ***
  % *** trigger bugs that do not appear when using other class files.       ***                          ***
  % The testflow support page is at:
  % http://www.michaelshell.org/tex/testflow/



  \documentclass[conference,compsoc]{IEEEtran}
  % Some/most Computer Society conferences require the compsoc mode option,
  % but others may want the standard conference format.
  %
  % If IEEEtran.cls has not been installed into the LaTeX system files,
  % manually specify the path to it like:
  % \documentclass[conference,compsoc]{../sty/IEEEtran}

  \usepackage{graphicx}
  \usepackage{epstopdf}
  \DeclareGraphicsExtensions{.eps}
  \usepackage{url}
  \usepackage{multicol}% http://ctan.org/pkg/multicols


  % Some very useful LaTeX packages include:
% (uncomment the ones you want to load)


  % *** MISC UTILITY PACKAGES ***
  %
  %\usepackage{ifpdf}
  % Heiko Oberdiek's ifpdf.sty is very useful if you need conditional
  % compilation based on whether the output is pdf or dvi.
  % usage:
  % \ifpdf
  %   % pdf code
  % \else
  %   % dvi code
  % \fi
  % The latest version of ifpdf.sty can be obtained from:
  % http://www.ctan.org/pkg/ifpdf
  % Also, note that IEEEtran.cls V1.7 and later provides a builtin
  % \ifCLASSINFOpdf conditional that works the same way.
  % When switching from latex to pdflatex and vice-versa, the compiler may
  % have to be run twice to clear warning/error messages.






  % *** CITATION PACKAGES ***
  %
  \ifCLASSOPTIONcompsoc
  % IEEE Computer Society needs nocompress option
% requires cite.sty v4.0 or later (November 2003)
  \usepackage[nocompress]{cite}
  \else
  % normal IEEE
  \usepackage{cite}
  \fi
  % cite.sty was written by Donald Arseneau
  % V1.6 and later of IEEEtran pre-defines the format of the cite.sty package
  % \cite{} output to follow that of the IEEE. Loading the cite package will
  % result in citation numbers being automatically sorted and properly
  % "compressed/ranged". e.g., [1], [9], [2], [7], [5], [6] without using
  % cite.sty will become [1], [2], [5]--[7], [9] using cite.sty. cite.sty's
  % \cite will automatically add leading space, if needed. Use cite.sty's
  % noadjust option (cite.sty V3.8 and later) if you want to turn this off
  % such as if a citation ever needs to be enclosed in parenthesis.
  % cite.sty is already installed on most LaTeX systems. Be sure and use
  % version 5.0 (2009-03-20) and later if using hyperref.sty.
  % The latest version can be obtained at:
  % http://www.ctan.org/pkg/cite
  % The documentation is contained in the cite.sty file itself.
  %
  % Note that some packages require special options to format as the Computer
  % Society requires. In particular, Computer Society  papers do not use
  % compressed citation ranges as is done in typical IEEE papers
  % (e.g., [1]-[4]). Instead, they list every citation separately in order
  % (e.g., [1], [2], [3], [4]). To get the latter we need to load the cite
  % package with the nocompress option which is supported by cite.sty v4.0
  % and later.





  % *** GRAPHICS RELATED PACKAGES ***
  %
  \ifCLASSINFOpdf
  % \usepackage[pdftex]{graphicx}
  % declare the path(s) where your graphic files are
  % \graphicspath{{../pdf/}{../jpeg/}}
  % and their extensions so you won't have to specify these with
  % every instance of \includegraphics
  % \DeclareGraphicsExtensions{.pdf,.jpeg,.png}
  \else
  % or other class option (dvipsone, dvipdf, if not using dvips). graphicx
  % will default to the driver specified in the system graphics.cfg if no
  % driver is specified.
  % \usepackage[dvips]{graphicx}
  % declare the path(s) where your graphic files are
  % \graphicspath{{../eps/}}
  % and their extensions so you won't have to specify these with
  % every instance of \includegraphics
  % \DeclareGraphicsExtensions{.eps}
  \fi
  % graphicx was written by David Carlisle and Sebastian Rahtz. It is
  % required if you want graphics, photos, etc. graphicx.sty is already
  % installed on most LaTeX systems. The latest version and documentation
  % can be obtained at: 
  % http://www.ctan.org/pkg/graphicx
  % Another good source of documentation is "Using Imported Graphics in
  % LaTeX2e" by Keith Reckdahl which can be found at:
  % http://www.ctan.org/pkg/epslatex
  %
  % latex, and pdflatex in dvi mode, support graphics in encapsulated
  % postscript (.eps) format. pdflatex in pdf mode supports graphics
  % in .pdf, .jpeg, .png and .mps (metapost) formats. Users should ensure
  % that all non-photo figures use a vector format (.eps, .pdf, .mps) and
  % not a bitmapped formats (.jpeg, .png). The IEEE frowns on bitmapped formats
  % which can result in "jaggedy"/blurry rendering of lines and letters as
  % well as large increases in file sizes.
  %
  % You can find documentation about the pdfTeX application at:
  % http://www.tug.org/applications/pdftex





  % *** MATH PACKAGES ***
  %
  %\usepackage{amsmath}
  % A popular package from the American Mathematical Society that provides
  % many useful and powerful commands for dealing with mathematics.
  %
  % Note that the amsmath package sets \interdisplaylinepenalty to 10000
  % thus preventing page breaks from occurring within multiline equations. Use:
  %\interdisplaylinepenalty=2500
  % after loading amsmath to restore such page breaks as IEEEtran.cls normally
  % does. amsmath.sty is already installed on most LaTeX systems. The latest
  % version and documentation can be obtained at:
  % http://www.ctan.org/pkg/amsmath





  % *** SPECIALIZED LIST PACKAGES ***
  %
  %\usepackage{algorithmic}
  % algorithmic.sty was written by Peter Williams and Rogerio Brito.
  % This package provides an algorithmic environment fo describing algorithms.
  % You can use the algorithmic environment in-text or within a figure
  % environment to provide for a floating algorithm. Do NOT use the algorithm
  % floating environment provided by algorithm.sty (by the same authors) or
  % algorithm2e.sty (by Christophe Fiorio) as the IEEE does not use dedicated
  % algorithm float types and packages that provide these will not provide
  % correct IEEE style captions. The latest version and documentation of
  % algorithmic.sty can be obtained at:
  % http://www.ctan.org/pkg/algorithms
% Also of interest may be the (relatively newer and more customizable)
  % algorithmicx.sty package by Szasz Janos:
  % http://www.ctan.org/pkg/algorithmicx




  % *** ALIGNMENT PACKAGES ***
  %
  %\usepackage{array}
  % Frank Mittelbach's and David Carlisle's array.sty patches and improves
  % the standard LaTeX2e array and tabular environments to provide better
  % appearance and additional user controls. As the default LaTeX2e table
  % generation code is lacking to the point of almost being broken with
  % respect to the quality of the end results, all users are strongly
% advised to use an enhanced (at the very least that provided by array.sty)
  % set of table tools. array.sty is already installed on most systems. The
  % latest version and documentation can be obtained at:
  % http://www.ctan.org/pkg/array


  % IEEEtran contains the IEEEeqnarray family of commands that can be used to
  % generate multiline equations as well as matrices, tables, etc., of high
  % quality.




  % *** SUBFIGURE PACKAGES ***
  %\ifCLASSOPTIONcompsoc
  %  \usepackage[caption=false,font=footnotesize,labelfont=sf,textfont=sf]{subfig}
  %\else
  %  \usepackage[caption=false,font=footnotesize]{subfig}
  %\fi
  % subfig.sty, written by Steven Douglas Cochran, is the modern replacement
  % for subfigure.sty, the latter of which is no longer maintained and is
  % incompatible with some LaTeX packages including fixltx2e. However,
  % subfig.sty requires and automatically loads Axel Sommerfeldt's caption.sty
  % which will override IEEEtran.cls' handling of captions and this will result
  % in non-IEEE style figure/table captions. To prevent this problem, be sure
  % and invoke subfig.sty's "caption=false" package option (available since
      % subfig.sty version 1.3, 2005/06/28) as this is will preserve IEEEtran.cls
  % handling of captions.
  % Note that the Computer Society format requires a sans serif font rather
  % than the serif font used in traditional IEEE formatting and thus the need
  % to invoke different subfig.sty package options depending on whether
  % compsoc mode has been enabled.
  %
  % The latest version and documentation of subfig.sty can be obtained at:
  % http://www.ctan.org/pkg/subfig




  % *** FLOAT PACKAGES ***
  %
  %\usepackage{fixltx2e}
  % fixltx2e, the successor to the earlier fix2col.sty, was written by
  % Frank Mittelbach and David Carlisle. This package corrects a few problems
  % in the LaTeX2e kernel, the most notable of which is that in current
  % LaTeX2e releases, the ordering of single and double column floats is not
  % guaranteed to be preserved. Thus, an unpatched LaTeX2e can allow a
  % single column figure to be placed prior to an earlier double column
  % figure.
  % Be aware that LaTeX2e kernels dated 2015 and later have fixltx2e.sty's
  % corrections already built into the system in which case a warning will
  % be issued if an attempt is made to load fixltx2e.sty as it is no longer
  % needed.
  % The latest version and documentation can be found at:
  % http://www.ctan.org/pkg/fixltx2e


  %\usepackage{stfloats}
  % stfloats.sty was written by Sigitas Tolusis. This package gives LaTeX2e
  % the ability to do double column floats at the bottom of the page as well
  % as the top. (e.g., "\begin{figure*}[!b]" is not normally possible in
      % LaTeX2e). It also provides a command:
  %\fnbelowfloat
  % to enable the placement of footnotes below bottom floats (the standard
      % LaTeX2e kernel puts them above bottom floats). This is an invasive package
  % which rewrites many portions of the LaTeX2e float routines. It may not work
  % with other packages that modify the LaTeX2e float routines. The latest
  % version and documentation can be obtained at:
  % http://www.ctan.org/pkg/stfloats
  % Do not use the stfloats baselinefloat ability as the IEEE does not allow
  % \baselineskip to stretch. Authors submitting work to the IEEE should note
  % that the IEEE rarely uses double column equations and that authors should try
  % to avoid such use. Do not be tempted to use the cuted.sty or midfloat.sty
  % packages (also by Sigitas Tolusis) as the IEEE does not format its papers in
  % such ways.
  % Do not attempt to use stfloats with fixltx2e as they are incompatible.
  % Instead, use Morten Hogholm'a dblfloatfix which combines the features
  % of both fixltx2e and stfloats:
  %
  % \usepackage{dblfloatfix}
  % The latest version can be found at:
  % http://www.ctan.org/pkg/dblfloatfix




  % *** PDF, URL AND HYPERLINK PACKAGES ***
  %
  %\usepackage{url}
  % url.sty was written by Donald Arseneau. It provides better support for
  % handling and breaking URLs. url.sty is already installed on most LaTeX
  % systems. The latest version and documentation can be obtained at:
  % http://www.ctan.org/pkg/url
  % Basically, \url{my_url_here}.




  % *** Do not adjust lengths that control margins, column widths, etc. ***
  % *** Do not use packages that alter fonts (such as pslatex).         ***
  % There should be no need to do such things with IEEEtran.cls V1.6 and later.
  % (Unless specifically asked to do so by the journal or conference you plan
      % to submit to, of course. )


  % correct bad hyphenation here
  \hyphenation{op-tical net-works semi-conduc-tor}

  \usepackage[utf8x]{inputenc} 

  \usepackage{array}
  \newcolumntype{L}[1]{>{\raggedright\let\newline\\\arraybackslash\hspace{0pt}}m{#1}}
  \newcolumntype{C}[1]{>{\centering\let\newline\\\arraybackslash\hspace{0pt}}m{#1}}
  \newcolumntype{R}[1]{>{\raggedleft\let\newline\\\arraybackslash\hspace{0pt}}m{#1}}

  \usepackage{float}
  \usepackage{listings}


  \begin{document}
  %
  % paper title
  % Titles are generally capitalized except for words such as a, an, and, as,
  % at, but, by, for, in, nor, of, on, or, the, to and up, which are usually
  % not capitalized unless they are the first or last word of the title.
  % Linebreaks \\ can be used within to get better formatting as desired.
  % Do not put math or special symbols in the title.
  \title{Introdução ao PERF}

  % author names and affiliations
  % use a multiple column layout for up to three different
  % affiliations
  \author{\IEEEauthorblockN{Filipe Oliveira}
    \IEEEauthorblockA{Departamento de Informática\\
      Universidade do Minho\\
        Email: a57816@alunos.uminho.pt}
  }

% conference papers do not typically use \thanks and this command
% is locked out in conference mode. If really needed, such as for
% the acknowledgment of grants, issue a \IEEEoverridecommandlockouts
% after \documentclass

% for over three affiliations, or if they all won't fit within the width
% of the page (and note that there is less available width in this regard for
    % compsoc conferences compared to traditional conferences), use this
% alternative format:
% 
%\author{\IEEEauthorblockN{Michael Shell\IEEEauthorrefmark{1},
  %Homer Simpson\IEEEauthorrefmark{2},
  %James Kirk\IEEEauthorrefmark{3}, 
  %Montgomery Scott\IEEEauthorrefmark{3} and
    %Eldon Tyrell\IEEEauthorrefmark{4}}
  %\IEEEauthorblockA{\IEEEauthorrefmark{1}School of Electrical and Computer Engineering\\
    %Georgia Institute of Technology,
  %Atlanta, Georgia 30332--0250\\ Email: see http://www.michaelshell.org/contact.html}
  %\IEEEauthorblockA{\IEEEauthorrefmark{2}Twentieth Century Fox, Springfield, USA\\
    %Email: homer@thesimpsons.com}
  %\IEEEauthorblockA{\IEEEauthorrefmark{3}Starfleet Academy, San Francisco, California 96678-2391\\
    %Telephone: (800) 555--1212, Fax: (888) 555--1212}
  %\IEEEauthorblockA{\IEEEauthorrefmark{4}Tyrell Inc., 123 Replicant Street, Los Angeles, California 90210--4321}}




  % use for special paper notices
  %\IEEEspecialpapernotice{(Invited Paper)}




  % make the title area
  \maketitle

  % As a general rule, do not put math, special symbols or citations
  % in the abstract
  %\begin{abstract}

  %Neste estudo, analisamos a performance de kernels 
  %\end{abstract}

  % no keywords




  % For peer review papers, you can put extra information on the cover
  % page as needed:
  % \ifCLASSOPTIONpeerreview
  % \begin{center} \bfseries EDICS Category: 3-BBND \end{center}
  % \fi
  %
  % For peerreview papers, this IEEEtran command inserts a page break and
  % creates the second title. It will be ignored for other modes.
  \IEEEpeerreviewmaketitle



  %\section{Introduction}
  % no \IEEEPARstart
  %This demo file is intended to serve as a ``starter file''
  %for IEEE Computer Society conference papers produced under \LaTeX\ using
  %IEEEtran.cls version 1.8b and later.
  % You must have at least 2 lines in the paragraph with the drop letter
% (should never be an issue)
  %I wish you the best of success.

  %\hfill Filipe Oliveira

  %\hfill 1 Março, 2016

  \section{Introdução -- Contextualização do PERF}

  \section{Contextualização das métricas de performance em estudo}

  \section{Caracterização do Hardware do ambiente de testes}
  
  Especificadas as métricas de performance em estudo, resta-nos especificar os ambientes de teste nos quais pretendemos realizar as benchmarks. \par 
  Através da análise do hardware disponível no Search6 \footnote{Services and Advanced Research Computing with HTC/HPC clusters}, uma das nossas plataformas de teste, foram seleccionados nós do tipo compute-431, e compute 651,sendo a disponibilidade global dos mesmos o principal factor. Nas tabelas \ref{table:characterization_431} e \ref{table:characterization_651} encontram-se especificadas as principais características dos sistemas em teste.\par 

  \begin{table}[H]
  \caption{Características de Hardware do nó 431}
  \label{table:characterization_search}
  \centering
  \begin{tabular}{ | l | r | }

  \hline
  Sistema & compute-431 \\ \hline \hline
  \# CPUs & 2  \\ \hline
  CPU & Intel\textsuperscript{\textregistered} Xeon\textsuperscript{\textregistered} X5650 \\ \hline 
  Arquitectura de Processador & Nehalem  \\ \hline 
  \# Cores por CPU & 6   \\ \hline 
  \# Threads por CPU & 12  \\ \hline 
  Freq. Clock & 2.66 GHz  \\ \hline
  Cache L1  & 192KB  (32KB por Core)  \\ \hline 
  Cache L2  & 1536KB (256KB por Core)  \\ \hline 
  Cache L3  & 12288KB (partilhada) \\ \hline 
  Ext. Inst. Set  & SSE4.2   \\ \hline 
  \#Memory Channels & 3 \\ \hline
  Memória Ram Disponível & 48GB \\ \hline
  Peak Memory BW Fab. CPU  & 32 GB/s \\ \hline
	
  \end{tabular}
  \end{table}


  \section{Determinação do tamanho dos datasets}

sqrt( (12288KB / 4 bytes ) / 3 ) = 1011 



  \section{Determinação do tempo médio necessário para criar e  terminar um fio de execução}
  \subsection{Nós compute-431}
  Por forma a calcular o tempo médio necessário para criar e terminar um fio de execução foi criado um kernel, que apenas realizava essas mesmas operações e registados os valores para os diferentes número de threads. A tabela \ref{table:search_create} apresenta a relação entre média e desvio padrão de criação/terminação para um diferente número de POSIX Threads para os nós do tipo compute-431.


\section{Parte I }



\section{Parte 2 }


  \begin{table}[H]
  \caption{Performance events (naive vs. interchange) para o nó compute-431}
  \label{table:search_events}
  \centering
  \begin{tabular}{ | l | r | r |   }

  \hline
  \# EVENT NAME	 & NAIVE  & INTERCHANGE \\ \hline 
  cpu-cycles  & 535187277  & 399561216         \\ \hline       
  instructions &       1044692763 &      1152237507      \\ \hline
  cache-references &      8196140 &    429971      \\ \hline
  cache-misses     &    36522 &     43034      \\ \hline
  branch-instructions & 126101720 & 132065934      \\ \hline
  branch-misses     &   258384 &  249858      \\ \hline
  bus-cycles       &       0 &    0      \\ \hline
   L1-dcache-loads  &  246027409 & 253077242     \\ \hline
 L1-dcache-load-misses & 56436199 & 7577858   \\ \hline
  L1-dcache-stores   &  9973628 & 128034804     \\ \hline
  L1-dcache-store-misses & 322982 &  106020     \\ \hline
  LLC-loads           &    7391770 & 262810     \\ \hline
  LLC-load-misses      &  2671 &  1001     \\ \hline
  LLC-stores          &  218407 & 69369     \\ \hline
  LLC-store-misses    &   18512 &    0     \\ \hline
  dTLB-load-misses    &   2239 & 950     \\ \hline
  dTLB-store-misses   &    446 &  9     \\ \hline
  iTLB-load-misses   &   0 &   0     \\ \hline
  branch-loads   & 129163483 & 129898962  \\ \hline    
  branch-load-misses &  5688441 & 5560030      \\ \hline
  \end{tabular}
  \end{table}
  
  
  
  \begin{table}[H]
  \caption{Performance events (naive vs. interchange) para o nó compute-431}
  \label{table:search_events}
  \centering
  \begin{tabular}{ | l | r | r |   }

  \hline
  RATIO OR RATE		 & NAIVE  & INTERCHANGE \\ \hline 
  Elapsed time (seconds) & 0.2041 & 0.1597   \\ \hline      
  
  Instructions per cycle  & 1.95  IPC & 2.88 IPC       \\ \hline      
  % Instructions per cycle	instructions / cycles
    % 1044692763 / 500890840 = 2,085
  % 1152237507 / 399561216 = 2,8837
  
          
  
  
  L1 cache miss ratio	  & 22,9389 \%   &     2,9942 \%    \\ \hline      
  % L1 cache miss ratio	L1-dcache-loads / L1-dcache-load-misses
  % 246027409 / 56436199  = 4,3593
  % 56436199 /  246027409 * 100 = 22,938988477
   % 253077242 / 7577858   = 33,3969
   
      % 7577858 / 253077242 * 100   = 2,9942866218

  L1 cache miss rate PTI &  54,0218 &     6,5766    \\ \hline     
  %L1 cache miss rate PTI 	L1-dcache-load-misses / (instructions / 1000)
     % 56436199 / ( 1044692763 / 1000)  = 54,0218
          %  (( 1044692763 / 1000)  / 56436199 ) *100 = 

   %  7577858 / ( 1152237507 / 1000)  = 6,5766
   
    L3 cache miss ratio	  & 0,0361 \%   &     0,3808 \%   \\ \hline      
  % L3 cache miss ratio	LLC-loads / LLC-load-misses
  % 7391770 / 2671  = 4,3593
    % (2671 / 7391770) * 100 = 0,0361

  % (1001 / 262810) * 100 = 0,3808
   % 1001 / 262810    = 33,3969
   
  
  Data TLB miss ratio	 &    0,00027 &   0,0022     \\ \hline      
  % Data TLB miss ratio	dTLB-load-misses / cache-references
  % 2239 / 8196140  = 0,00027
   %  950 / 429971  = 0,0008
   
  Data TLB miss rate PTI	 &   0,0021 &   0,0008    \\ \hline      
  % Data TLB miss rate PTI	dTLB-load-misses / (instructions / 1000)
   % 2239 / ( 1044692763 / 1000)  = 0,0021
   %  950 / ( 1152237507 / 1000)  = 0,0008
  
  Branch mispredict ratio	 & 0,002   &     0,0019  \\ \hline      
  % Branch mispredict ratio	branch-misses / branch-instructions
  % 258384 /  126101720 = 0,002
    % 249858 /  132065934 = 0,0019

  
  Branch mispredict rate PTI	 &  0,2473   &   0,2168    \\ \hline          
  % Branch mispredict rate PTI	branch-misses / (instructions / 1000)
  % 258384 / ( 1044692763 / 1000)  = 0,2473
   %  249858 / ( 1152237507 / 1000)  = 0,2168
   
     \end{tabular}
  \end{table}
  
  
  
  \section{Parte 3 }

\subsection{Período de amostragem e frequência de amostragem}

Dado que pretendemos encontrar os hotspots de ambas as versões, iremos usar a amostragem baseada em cpu-cycles. Desta forma, porções da aplicação que consumam mais tempo terão um maior número de amostras registadas.
Analisemos o tempo que demoram as versões large\_naive e large\_interchange sem qualquer tipo de ferramenta de amostragem. Denote que por forma a validar os resultados foram realizadas 50 medições, sendo o valor apresentado o K Best (sendo K = 3):

 \begin{tabular}{ | l | r | r |   }

  \hline
  \# EVENT NAME	 & LARGE NAIVE  & LARGE INTERCHANGE \\ \hline 
   Elapsed time & 10.460 &  \\ \hline    
  \end{tabular}




Cada máquina tem um numero de contadores maximos e fixados a um certo tipo de eventos. Multiplexagem gasta tempo. 
Peso muito grande na avaliação de desempenho. quais os contadores do nosso interesse e a frequencia de amostragem. 


When looking for the hottest spots in an application, you may choose to profile your program using cpu-cycles because the cpu-cycles event is a measure of time just like the cpu-clock software event.


 \begin{table}[H]
  \caption{Performance events (naive vs. interchange) para o nó compute-431}
  \label{table:search_events}
  \centering
  \begin{tabular}{ | l | r | r |   }

  \hline
  \# EVENT NAME	 & NAIVE  & INTERCHANGE \\ \hline 
   Elapsed time & &  \\ \hline    
  instructions	& 38376000000 &  41776400000 \\ \hline    
cycles	& 78390700000 &  12384800000 \\ \hline    
cache-references	& 4744200000 &  14400000 \\ \hline    
cache-misses	&  4008300000 &  11200000 \\ \hline    
LLC-loads	& 4815000000  &  14800000 \\ \hline    
LLC-load-misses	& 4073600000 & 14400000  \\ \hline    
dTLB-load-misses & 1100000	& 100000  \\ \hline    
branches	& 3923900000 & 3847500000  \\ \hline    
branch-misses	&  2200000 &  1900000 \\ \hline    

  \end{tabular}
  \end{table}

\subsection{Análise comparativa do número de amostras por evento de hardware para as versões large\_naive e large\_interchange}

Da análise de execução para as versões large\_naive e large\_interchange, 
 large_naive and large_interchange, and collected event sample data for the most relevant hardware events. A comparative summary is shown in the following table.


 \begin{table}[H]
  \caption{Sampling mode: large\_naive vs. large\_interchange
 para o nó compute-431}
  \label{table:search_sampling}
  \centering
  \begin{tabular}{ | l | r | r |   }

  \hline
  \# EVENT NAME	 & NAIVE  & INTERCHANGE \\ \hline   
  Elapsed time & &  \\ \hline    
  instructions & 383K samples	& 417K samples  \\ \hline    
cycles	& 783K  samples &  123K samples  \\ \hline    
cache-references &	47K samples & 144 samples   \\ \hline    
cache-misses & 40K samples	& 112 samples \\ \hline    
LLC-loads	 &  48K samples &  148 samples \\ \hline    
LLC-load-misses	& 40K samples & 144 samples  \\ \hline    
dTLB-load-misses	& 11 samples & 1 samples \\ \hline    
branches	&  39K samples &  38K samples \\ \hline    
branch-misses	& 22 samples & 19 samples \\ \hline    
   \end{tabular}
  \end{table}



  \section{Conclusão}






  % conference papers do not normally have an appendix



  % use section* for acknowledgment
  %\ifCLASSOPTIONcompsoc
  % The Computer Society usually uses the plural form
  % \section*{Acknowledgments}
  %\else
  % regular IEEE prefers the singular form
  %\section*{Acknowledgment}
  %\fi


  %The authors would like to thank...





  % trigger a \newpage just before the given reference
  % number - used to balance the columns on the last page
  % adjust value as needed - may need to be readjusted if
  % the document is modified later
  %\IEEEtriggeratref{8}
  % The "triggered" command can be changed if desired:
  %\IEEEtriggercmd{\enlargethispage{-5in}}

  % references section

  % can use a bibliography generated by BibTeX as a .bbl file
  % BibTeX documentation can be easily obtained at:
  % http://mirror.ctan.org/biblio/bibtex/contrib/doc/
  % The IEEEtran BibTeX style support page is at:
  % http://www.michaelshell.org/tex/ieeetran/bibtex/
  %\bibliographystyle{IEEEtran}
% argument is your BibTeX string definitions and bibliography database(s)
  %\bibliography{IEEEabrv,../bib/paper}
  %
  % <OR> manually copy in the resultant .bbl file
  % set second argument of \begin to the number of references
% (used to reserve space for the reference number labels box)


  %\begin{thebibliography}{1}

  %\%bibitem{nas}
  %NAS Parallel Benchmarks, \url{http://www.nas.nasa.gov/publications/npb.html}

  %\end{thebibliography}

  %\appendix 


  % that's all folks
  \end{document}


